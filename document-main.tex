% -*- coding: UTF-8 -*-
% 青岛工学院毕业论文(设计)撰写---范例
% 以青岛那个工学院毕业论文撰写范例作为模板,展示如何使用LaTeX撰写毕业论文设计文档等正式文档。
% 作为这个模板的撰写者,也是青岛工学院的准毕业生,每年看到不少人因为论文格式不够规范、或word版本不兼容、字体不兼容等原因导致多次返工。
% 在接触到LaTeX的时候就萌生了一个想法,就是利用这门排版语言编写一套毕业论文模板。
% 此举不光是为了为后人提供便利,也为了推动LaTeX在青岛工学院中的使用出一份力量。
% 希望在做成这套模板之后,能有青工学子加入进来,共同维护这个模板,让更多的人不再为论文格式而发愁,将精力放在论文内容中。

\documentclass[fontset=adobe, twoside, a4paper]{oucqdc-document}
%\usepackage{ctex}
\makeatletter
\newcommand\unline[2][4cm]{\hskip1pt\underline{\hb@xt@ #1{\hss#2\hss}}\hskip3pt}
\makeatother
%
% 今日已解决 模板目录规划
% 今日待解决问题 标题页的版面设计(标题字体,页边距,单面打印<默认>)
%              使用titling包对标题格式进行修改(去掉日期,想办法把签字区域加入标题页)
%

\begin{document}
\begin{center}
\xiaoerhao\heiti 青岛工学院毕业论文(设计)题目 \\
\xiaosihao\songti(题目,小二号,黑体)
\begin{figure}[!b]
\setlength{\baselineskip}{35pt}
\xiaosanhao
\begin{tabbing}
 \hspace{5cm}完成日期:\hspace{2cm} \= \unline[6cm]{} \\
 \hspace{5cm}指导教师签字: \> \unline[6cm]{} \\
 \hspace{5cm}评阅教师签字: \> \unline[6cm]{} \\
 \hspace{5cm}答辩小组长签字: \> \unline[6cm]{} \\
 \hspace{5cm}答辩小组成员签字: \> \unline[6cm]{} \\ \\ \\ 
\end{tabbing}
\end{figure}
\end{center}
\end{document}
